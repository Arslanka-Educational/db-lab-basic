\documentclass{article}

\usepackage[utf8]{inputenc}
\usepackage[russian]{babel}
\usepackage[a4paper, margin=1in]{geometry}
\usepackage{graphicx}
\usepackage{amsmath}
\usepackage{wrapfig}
\usepackage{multirow}
\usepackage{mathtools}
\usepackage{pgfplots}
\usepackage{pgfplotstable}
\usepackage{setspace}
\usepackage{changepage}
\usepackage{caption}
\usepackage{csquotes}
\usepackage{hyperref}
\usepackage{listings}

\pgfplotsset{compat=1.18}
\hypersetup{
  colorlinks = true,
  linkcolor  = blue,
  filecolor  = magenta,      
  urlcolor   = darkgray,
  pdftitle   = {
    database-report-3-arslan-gin
  },
}
  
\definecolor{codegreen}{rgb}{0,0.6,0}
\definecolor{codegray}{rgb}{0.5,0.5,0.5}
\definecolor{codepurple}{rgb}{0.58,0,0.82}
\definecolor{backcolour}{rgb}{0.99,0.99,0.99}

\lstdefinestyle{codestyle}{
  backgroundcolor=\color{backcolour},   
  commentstyle=\color{codegreen},
  keywordstyle=\color{magenta},
  numberstyle=\tiny\color{codegray},
  stringstyle=\color{codepurple},
  basicstyle=\ttfamily\footnotesize,
  breakatwhitespace=false,         
  breaklines=true,                 
  captionpos=b,                    
  keepspaces=true,                 
  numbers=left,                    
  numbersep=5pt,                  
  showspaces=false,                
  showstringspaces=false,
  showtabs=false,                  
  tabsize=2
}

\graphicspath{ {./img/} }

\begin{document}

\begin{titlepage}
    \begin{center}
        \begin{spacing}{1.4}
            \large{Университет ИТМО} \\
            \large{Факультет программной инженерии и компьютерной техники} \\
        \end{spacing}
        \vfill
        \textbf{
            \huge{Базы данных.} \\
            \huge{Лабораторная работа №3.} \\
        }
    \end{center}
    \vfill
    \begin{center}
        \begin{tabular}{r l}
            Группа:  & P33121                  \\
            Студент: & Гиниятуллин Арслан Рафаилович \\
            Вариант: & 44949
        \end{tabular}
    \end{center}
    \vfill
    \begin{center}
        \begin{large}
            2023
        \end{large}
    \end{center}
\end{titlepage}

\section*{Ключевые слова}

База данных, PostgreSQL,
даталогическая модель,
инфологическая модель.

\tableofcontents

\section{Цель работы}

Научиться выполнять сложные SQL
запросы.

\section{Текст задания}

\subsection{Задание 1}
Сделать запрос для получения атрибутов из указанных таблиц, применив фильтры по указанным условиям: \\ \\

\textbf{Таблицы:} Н\_ОЦЕНКИ, Н\_ВЕДОМОСТИ. \\
\textbf{Вывести атрибуты}: Н\_ОЦЕНКИ.ПРИМЕЧАНИЕ, Н\_ВЕДОМОСТИ.ИД. \\
\textbf{Фильтры (AND)}:
\begin{enumerate}
    \item Н\_ОЦЕНКИ.КОД = 4.
    \item Н\_ВЕДОМОСТИ.ДАТА < 2010-06-18.
\end{enumerate}

\textbf{Вид соединения:} LEFT JOIN.

\subsection{Задание 2}
Сделать запрос для получения атрибутов из указанных таблиц, применив фильтры по указанным условиям:  \\ \\

\textbf{Таблицы:} Н\_ЛЮДИ, Н\_ОБУЧЕНИЯ, Н\_УЧЕНИКИ. \\
\textbf{Вывести атрибуты:} Н\_ЛЮДИ.ФАМИЛИЯ, Н\_ОБУЧЕНИЯ.НЗК, Н\_УЧЕНИКИ.ИД. \\
\textbf{Фильтры: (AND)}
\begin{enumerate}
    \item Н\_ЛЮДИ.ОТЧЕСТВО > Георгиевич.
    \item Н\_ОБУЧЕНИЯ.НЗК = 001000.
    \item Н\_УЧЕНИКИ.ИД < 100410.
\end{enumerate}

\textbf{Вид соединения:} RIGHT JOIN.

\subsection{Задание 3}
Вывести число студентов ФКТИУ, которые без ИНН. \\
Ответ должен содержать только одно число.

\subsection{Задание 4}
Выдать различные отчества студентов и число людей с каждой из этих отчеств, ограничив список отчествами, встречающимися менее 10 раз на на заочной форме обучения. \\
Для реализации использовать подзапрос.

\subsection{Задание 5}
Выведите таблицу со средним возрастом студентов во всех группах (Группа, Средний возраст), где средний возраст равен максимальному возрасту в группе 1101.

\subsection{Задание 6}
Получить список студентов, зачисленных после первого сентября 2012 года на первый курс очной или заочной формы обучения. \\
В результат включить:
\begin{itemize}
    \item номер группы;
    \item номер, фамилию, имя и отчество студента;
    \item номер и состояние пункта приказа;
\end{itemize}
\\
Для реализации использовать подзапрос с EXISTS.


\subsection{Задание 7}
Вывести список студентов, имеющих одинаковые фамилии, но не совпадающие даты рождения.

\section{SQL-запросы}

\subsection{Запрос 1}
\begin{verbatim}
SELECT 
  "Н_ОЦЕНКИ"."ПРИМЕЧАНИЕ", 
  "Н_ВЕДОМОСТИ"."ИД" 
FROM 
  ucheb.public."Н_ОЦЕНКИ" 
  LEFT OUTER JOIN ucheb.public."Н_ВЕДОМОСТИ" ON "Н_ОЦЕНКИ"."КОД" = "Н_ВЕДОМОСТИ"."ОЦЕНКА" 
WHERE 
  "Н_ОЦЕНКИ"."КОД" = '4' 
  AND "Н_ВЕДОМОСТИ"."ДАТА" < '2010-06-18';
\end{verbatim}

\subsection{Запрос 2}
\begin{verbatim}
SELECT
    "Н_ЛЮДИ"."ФАМИЛИЯ",
    "Н_ОБУЧЕНИЯ"."НЗК",
    "Н_УЧЕНИКИ"."ИД"
FROM
    "Н_ЛЮДИ"
        RIGHT OUTER JOIN "Н_ОБУЧЕНИЯ" ON "Н_ЛЮДИ"."ИД" = "Н_ОБУЧЕНИЯ"."ЧЛВК_ИД"
        RIGHT OUTER JOIN "Н_УЧЕНИКИ" ON "Н_ОБУЧЕНИЯ"."ЧЛВК_ИД" = "Н_УЧЕНИКИ"."ЧЛВК_ИД"
WHERE
        "Н_ЛЮДИ"."ОТЧЕСТВО" > 'Георгиевич'
  AND "Н_ОБУЧЕНИЯ"."НЗК" = '001000'
  AND "Н_УЧЕНИКИ"."ИД" < 100410;
\end{verbatim}

\subsection{Запрос 3} 
\begin{verbatim}
SELECT 
  COUNT("Н_УЧЕНИКИ"."ИД") 
FROM 
  "Н_УЧЕНИКИ" 
  RIGHT OUTER JOIN "Н_ЛЮДИ" ON "Н_УЧЕНИКИ"."ЧЛВК_ИД" = "Н_ЛЮДИ"."ИД" 
  INNER JOIN "Н_ПЛАНЫ" ON "Н_УЧЕНИКИ"."ПЛАН_ИД" = "Н_ПЛАНЫ"."ИД" 
  INNER JOIN "Н_ОТДЕЛЫ" ON "Н_ПЛАНЫ"."ОТД_ИД" = "Н_ОТДЕЛЫ"."ИД" 
WHERE 
  "Н_ОТДЕЛЫ"."КОРОТКОЕ_ИМЯ" = 'ФКТИУ' 
  AND "Н_ЛЮДИ"."ИНН" IS NULL;
\end{verbatim}

\subsection{Запрос 4} 
\begin{verbatim}
    SELECT 
  "Н_ЛЮДИ"."ОТЧЕСТВО", 
  COUNT(
    "Н_ЛЮДИ"."ОТЧЕСТВО"
  ) AS PAT_NAME_CNT 
FROM 
  "Н_ЛЮДИ" 
WHERE 
  10 > (
    SELECT 
      COUNT(a1."ИД") 
    FROM 
      "Н_ЛЮДИ" AS a1 
    WHERE 
      a1."ОТЧЕСТВО" = "Н_ЛЮДИ"."ОТЧЕСТВО"
  ) 
GROUP BY 
  "Н_ЛЮДИ"."ОТЧЕСТВО";
\end{verbatim}

\subsection{Запрос 5}
\begin{verbatim}
    SELECT
    "Н_УЧЕНИКИ"."ГРУППА",
    ROUND(AVG(
            EXTRACT(
                    year
                    FROM
                    age(
                        "Н_ЛЮДИ"."ДАТА_РОЖДЕНИЯ"
                    )
                )::int
        )) as AVG_AGE
FROM
    "Н_УЧЕНИКИ"
        RIGHT OUTER JOIN "Н_ЛЮДИ" ON "Н_УЧЕНИКИ"."ЧЛВК_ИД" = "Н_ЛЮДИ"."ИД"
WHERE  "Н_ЛЮДИ"."ДАТА_СМЕРТИ" > now()
GROUP BY
    "Н_УЧЕНИКИ"."ГРУППА"
HAVING ROUND(AVG(
        EXTRACT(
                year
                FROM
                age(
                        "Н_ЛЮДИ"."ДАТА_РОЖДЕНИЯ"
                    )
            )::int
    )) = (SELECT   MAX(
                             EXTRACT(
                                     year
                                     FROM
                                     age(
                                             "Н_ЛЮДИ"."ДАТА_РОЖДЕНИЯ"
                                         )
                                 )::int
                         ) as MX_AGE
          FROM
                     "Н_УЧЕНИКИ"
                         RIGHT OUTER JOIN "Н_ЛЮДИ" ON "Н_УЧЕНИКИ"."ЧЛВК_ИД" = "Н_ЛЮДИ"."ИД"
          WHERE  "Н_ЛЮДИ"."ДАТА_СМЕРТИ" > now() AND "Н_УЧЕНИКИ"."ГРУППА" = '1101');
\end{verbatim}


\subsection{Запрос 6}
\begin{verbatim}
SELECT
    "Н_УЧЕНИКИ"."ГРУППА",
    "Н_ЛЮДИ"."ИД",
    "Н_ЛЮДИ"."ИМЯ",
    "Н_ЛЮДИ"."ФАМИЛИЯ",
    "Н_ЛЮДИ"."ОТЧЕСТВО",
    "Н_УЧЕНИКИ"."ИД",
    "Н_УЧЕНИКИ"."СОСТОЯНИЕ"
FROM "Н_УЧЕНИКИ"
         RIGHT JOIN "Н_ЛЮДИ" ON "Н_УЧЕНИКИ"."ЧЛВК_ИД" = "Н_ЛЮДИ"."ИД"
         JOIN "Н_ПЛАНЫ" ON "Н_УЧЕНИКИ"."ПЛАН_ИД" = "Н_ПЛАНЫ"."ИД"
         JOIN "Н_ФОРМЫ_ОБУЧЕНИЯ" ON "Н_ПЛАНЫ"."ФО_ИД" = "Н_ФОРМЫ_ОБУЧЕНИЯ"."ИД"
WHERE "Н_ФОРМЫ_ОБУЧЕНИЯ"."НАИМЕНОВАНИЕ" IN ('Очная', 'Заочная')
  AND "Н_ПЛАНЫ"."КУРС" = 1
  AND "Н_УЧЕНИКИ"."НАЧАЛО" < '2012-09-01'::date;
\end{verbatim}

\subsection{Запрос 7}
\begin{verbatim}
WITH STUDENTS AS (
  SELECT 
    * 
  FROM 
    "Н_ЛЮДИ" 
    LEFT JOIN "Н_УЧЕНИКИ" ON "Н_ЛЮДИ"."ИД" = "Н_УЧЕНИКИ"."ЧЛВК_ИД"
) 
SELECT 
  S1."ФАМИЛИЯ", 
  S1."ИМЯ", 
  S1."ОТЧЕСТВО", 
  S1."ДАТА_РОЖДЕНИЯ", 
  S2."ФАМИЛИЯ", 
  S2."ИМЯ", 
  S2."ОТЧЕСТВО", 
  S2."ДАТА_РОЖДЕНИЯ" 
FROM 
  STUDENTS AS S1 
  JOIN STUDENTS AS S2 ON S1."ФАМИЛИЯ" = S2."ФАМИЛИЯ" 
WHERE 
  S1."ДАТА_РОЖДЕНИЯ" != S2."ДАТА_РОЖДЕНИЯ" 
  AND S1."ФАМИЛИЯ" != '.' 
  AND S2."ФАМИЛИЯ" != '.';
\end{verbatim}
\end{document}