\documentclass{article}

\usepackage[utf8]{inputenc}
\usepackage[russian]{babel}
\usepackage[a4paper, margin=1in]{geometry}
\usepackage{graphicx}
\usepackage{amsmath}
\usepackage{wrapfig}
\usepackage{multirow}
\usepackage{mathtools}
\usepackage{pgfplots}
\usepackage{pgfplotstable}
\usepackage{setspace}
\usepackage{changepage}
\usepackage{caption}
\usepackage{csquotes}
\usepackage{hyperref}
\usepackage{listings}

\pgfplotsset{compat=1.18}
\hypersetup{
  colorlinks = true,
  linkcolor  = blue,
  filecolor  = magenta,      
  urlcolor   = darkgray,
  pdftitle   = {
    database-report-1-arslan-gin
  },
}
  
\definecolor{codegreen}{rgb}{0,0.6,0}
\definecolor{codegray}{rgb}{0.5,0.5,0.5}
\definecolor{codepurple}{rgb}{0.58,0,0.82}
\definecolor{backcolour}{rgb}{0.99,0.99,0.99}

\lstdefinestyle{codestyle}{
  backgroundcolor=\color{backcolour},   
  commentstyle=\color{codegreen},
  keywordstyle=\color{magenta},
  numberstyle=\tiny\color{codegray},
  stringstyle=\color{codepurple},
  basicstyle=\ttfamily\footnotesize,
  breakatwhitespace=false,         
  breaklines=true,                 
  captionpos=b,                    
  keepspaces=true,                 
  numbers=left,                    
  numbersep=5pt,                  
  showspaces=false,                
  showstringspaces=false,
  showtabs=false,                  
  tabsize=2
}

\graphicspath{ {./img/} }

\lstset{style=codestyle}
\lstnewenvironment{code}[1][]%
  {\noindent\minipage{\linewidth}\medskip 
   \lstset{basicstyle=\ttfamily\footnotesize,frame=single,#1}}
  {\endminipage}

\begin{document}

\begin{titlepage}
    \begin{center}
        \begin{spacing}{1.4}
            \large{Университет ИТМО} \\
            \large{Факультет программной инженерии и компьютерной техники} \\
        \end{spacing}
        \vfill
        \textbf{
            \huge{Базы данных.} \\
            \huge{Лабораторная работа №1-2.} \\
        }
    \end{center}
    \vfill
    \begin{center}
        \begin{tabular}{r l}
            Группа:  & P33121                  \\
            Студент: & Гиниятуллин Арслан Рафаилович \\
            Вариант: & 5089
        \end{tabular}
    \end{center}
    \vfill
    \begin{center}
        \begin{large}
            2023
        \end{large}
    \end{center}
\end{titlepage}

\section*{Ключевые слова}

База данных, PostgreSQL,
даталогическая модель,
инфологическая модель.

\tableofcontents

\section{Цель работы}

Научиться проектировать базы данных,
составлять инфологические и
даталогические модели данных,
реализовывать их в БД
PostgreSQL, научиться выполнять
запросы.

\section{Текст задания}

В 7.00 он приносил Пулу в рубку управления тюбик кофе из кухни и сменял его. Если докладывать было не о чем и никаких срочных мер не требовалось (а так обычно и бывало), он начинал считывать показания приборов, а затем проводил всякие пробы и замеры, проверяя, нормально ли работают все агрегаты корабля. К 10.00 он обычно завершал проверку и переходил к занятиям.


\section{Описание предметной области}
Действия текста из задания происходит с моряками на корабле, поэтому выделим две стержневые сущности: \textbf{sailor} и \textbf{ship}. У каждого моряка есть квалификация \textbf{profession} и соответствующие виды деятельности \textbf{responsibility}. Например, одной из важных задач является считывание показаний приборов и проведение замеров. Замеры будем считывать с каких-то агрегатов корабля \textbf{ship\_component} и записывать в журнал \textbf{ship\_component\_measurement}. Также будем вести запись действий \textbf{action}, которые совершает моряк в конкретном месте \textbf{location}.
\section{Классификация сущностей}

\begin{enumerate}
    \item{\textbf{ship}} -- стержневая сущность
    \item{\textbf{sailor}}-- стержневая сущность
    \item{\textbf{profession}} -- характеристическая сущность
    \item{\textbf{ship\_component\_measurement}} -- характеристическая сущность
    \item{\textbf{action}} -- характеристическая сущность
    \item{\textbf{location}} -- характеристическая сущность
    \item{\textbf{ship\_component}} -- характеристическая сущность
    \item{\textbf{responsibility}} -- ассоциативная сущность
\end{enumerate}

\section{Инфологическая модель}
\begin{code}
    erDiagram
    ship |o--|{ ship_component: contains
    ship {
        id id
        name name
        manufacture_year manufacture_year
    }
    
    sailor }o--o| profession: serve
    sailor }o--|| location: located
    sailor ||--o{ ship_component_measurement: measure
    sailor {
        id id
        first_name first_name
        last_name last_name
        gender gender
        profession_id profession_id
        location_id location_id
    }
    
    profession }o--|{ responsibility: requires
    profession {
        id id
        name name
    }
    
    responsibility {
        id id
        name name
    }
    
    location {
        id id
        name name
        latitude latitude
        longitude longitude
    }
    
    action ||--|{ sailor: done_by
    action {
        id id
        name name
        description description
        start_time start_time
        duration duration
        sailor_id sailor_id
    }
    
    ship_component {
        id id
        name name
        serial_number serial_number
        ship_id ship_id
    }
    
    ship_component_measurment }o--|| ship_component: consist
    ship_component_measurment {
        name name
        result result
        unit unit
        sailor_id sailor_id
        ship_component_id ship_component_id
    }
\end{code}
\begin{figure}[ht]
    \includegraphics[scale=0.5]{./high-er-diagram.png}
    \caption{ERD infological from mermaid.live}
\centering
\end{figure}

\section{Даталогическая модель}
\begin{code}
erDiagram
    ship |o--|{ ship_component: contains
    ship {
        int id PK
        varchar(100) name
        date manufacture_year
    }
    
    sailor }o--o| profession: serve
    sailor }o--|| location: located
    sailor ||--o{ ship_component_measurment: measure
    sailor {
        int id PK
        varchar(50) first_name
        varchar(50) last_name
        enum_gender gender
        int profession_id FK
        int location_id FK
    }
    
    profession ||--|{ profession_responsibility: requires
    profession {
        int id PK
        varchar(100) name
    }
    
    profession_responsibility {
        int profession_id FK
        int responsibility_id FK
        pair_int profession_id_responsibility_id PK
    }
    
    responsibility ||--|{ profession_responsibility: commits
    responsibility {
        int id PK
        varchar(100) name
        text description
    }
    
    location {
        int id PK
        varchar(250) name
        decimal(10)8 latitude
        decimal(10)8 longitude
    }
    
    action ||--|{ sailor: done_by
    action {
        int id PK
        varchar(50) name
        text description
        timestamp start_time
        interval duration
        int sailor_id FK
    }
    ship_component {
        int id PK
        varchar(100) name
        varchar(50) serial_number
        int ship_id FK
    }
    
    ship_component_measurment }o--|| ship_component: consist
v REuni

s
s
ship_component_measurment {
        varchar(50) name
        real result
        enum_unit unit
        int sailor_id FK
        int ship_component_id FK
    }    
\end{code}
\begin{figure}[th]
    \includegraphics[scale=0.5]{./low-er-diagram.png}
    \caption{ERD datalogical from mermaid.live}
    \centering
\end{figure}

\section{Функциональные зависимости}
\begin{code}
    ship:
        id -> name
        id -> manufacture_year

    sailor:
        id -> first_name
        id -> last_name
        id -> gender
        id -> profession_id
        id -> location_id

    profession: 
        id -> name

    responsibility:
        id -> name
        id -> description

    location:
        id -> name
        id -> latitude
        id -> longitude

    action:
        id -> name
        id -> description
        id -> start_time
        id -> duration
        id -> sailor_id
        
    ship_component:
        id -> name
        id -> serial_number
        id -> ship_id
\end{code}

\section{1 Нормальая форма.}
    Отношение находится в 1 НФ, так как в каждом столбце не более одного аттрибута. Нет идентичных столбцов.
    
\section{2 Нормальная форма.}
    Отношение находится в 2 НФ, так как аттрибуты зависят целиком от PK, а не от его части. В данных переменных отношения ключи либо суррогатные и простые, либо естественные, покрывающие всю таблицу.

\section{3 Нормальная форма.}
    Отношение находится в 3 НФ, так как нет транзитивных функциональных зависимостей между аттрибутами: аттрибуты зависят исключительно от PK, но не друг от друга.

\section{Нормальная форма Бойса-Кодда.}
    Отношение находится в БКНФ, так как всякий детерминант является потенциальным ключом отношения: нет зависимости неключевых аттрибутов от ключевых

\section{Денормализация.}
    Можем объединить таблицы \textbf{ship\_component\_measurement} и \textbf{ship\_component}. Тогда при записи измерений об одном и том же агрегате придется дублировать информацию в таблице \textbf{ship\_component\_measurement}, что ведет к нарушению 3 НФ, так как аттрибуты таблицы \textbf{ship\_component} не связаны с результатом измерений. Если же, например объединить \textbf{profession} и \textbf{responsibility}, добавив массив аттрибутов ответственностей в колонку \textbf{responsibilities}, нарушим 1 НФ.
\end{document}
